\documentclass{article}
\usepackage[english]{babel}
\begin{document}

\section{Introduction}
\newtheorem{theorem}{Theorem}[section]
\newtheorem{corollary}{Corollary}[theorem]
\newtheorem{lemma}[theorem]{Lemma}

\begin{theorem}
Let \(f\) be a function whose derivative exists in every point, then \(f\) is a continuous function.
\end{theorem}

\begin{theorem}[Pythagorean theorem]
\label{pythagorean}	
This is about right triangles and can be summarised in the next equation \[ x^2 + y^2 = z^2 \]
\end{theorem}

And a consequence \ref{pythagorean} is the following statement:

\begin{corollary}
There's no right rectangle whose sides measure 3cm, 4cm, and 6cm.
\end{corollary}

\begin{lemma}
Given two line segments whose lengths are \(a\) and \(b\) respectively there is a real number \(r\) such that \(b=ra\).
\end{lemma}

\end{document}

% Source: This text was taken direclty from the Overleaf page! (Some changes were made to the original text.)